\documentclass[a4paper]{article}




\usepackage[margin=1in]{geometry} % full-width

% AMS Packages
\usepackage{amsmath}
\usepackage{amsthm}
\usepackage{amssymb}

% Unicode
\usepackage[utf8]{inputenc}
\usepackage{hyperref}
\hypersetup{
	unicode,
%	colorlinks,
%	breaklinks,
%	urlcolor=cyan, 
%	linkcolor=blue, 
	pdfauthor={Author One, Author Two, Author Three},
	pdftitle={A simple article template},
	pdfsubject={A simple article template},
	pdfkeywords={article, template, simple},
	pdfproducer={LaTeX},
	pdfcreator={pdflatex}
}

% Vietnamese
%\usepackage{vntex}

% Natbib
\usepackage[sort&compress,numbers,square]{natbib}
\bibliographystyle{mplainnat}

% Theorem, Lemma, etc
\theoremstyle{plain}
\newtheorem{theorem}{Theorem}
\newtheorem{corollary}[theorem]{Corollary}
\newtheorem{lemma}[theorem]{Lemma}
\newtheorem{claim}{Claim}[theorem]
\newtheorem{axiom}[theorem]{Axiom}
\newtheorem{conjecture}[theorem]{Conjecture}
\newtheorem{fact}[theorem]{Fact}
\newtheorem{hypothesis}[theorem]{Hypothesis}
\newtheorem{assumption}[theorem]{Assumption}
\newtheorem{proposition}[theorem]{Proposition}
\newtheorem{criterion}[theorem]{Criterion}
\theoremstyle{definition}
\newtheorem{definition}[theorem]{Definition}
\newtheorem{example}[theorem]{Example}
\newtheorem{remark}[theorem]{Remark}
\newtheorem{problem}[theorem]{Problem}
\newtheorem{principle}[theorem]{Principle}

\usepackage{graphicx, color}
\graphicspath{{fig/}}

%\usepackage[linesnumbered,ruled,vlined,commentsnumbered]{algorithm2e} % use algorithm2e for typesetting algorithms
\usepackage{algorithm, algpseudocode} % use algorithm and algorithmicx for typesetting algorithms
\usepackage{mathrsfs} % for \mathscr command



% Author info

\begin{document}
    \begin{titlepage}
        \centering
        {\includegraphics[width=0.2\textwidth]{logo}\par}
        \vspace{1cm}
        {\bfseries\LARGE Universidad Castilla La Mancha \par}
        \vspace{1cm}
        {\scshape\Large Ingeniería informática \par}
        \vspace{3cm}
        {\scshape\Huge linear data structure \par}
        \vspace{3cm}
        {\itshape\Large Data Structure laboratory\par}
        \vfill
        {\Large Authors: \par}
        {\Large Manuel Avilés Rodriguez  \par Ángel Garcia Collado\par Duygu Bayar\par María Jesús Dueñas Recuero }
        \vfill
        {\Large October 2021 \par}
    \end{titlepage}
	

\newpage
	\tableofcontents
	\newpage
	\section{Stack}
        \subsection{Problem description}
        There is a data file containing natural numbers. This file will be read number by number and you must apply the following treatment:
        \begin{itemize}
            \item Place the first number on a first stack.
            \item The following numbers will be compared with the one on top of the first stack. When the addition of the read number with the number on top is nine, you must insert two times the lower of the two on a second stack and remove the number on top of the first stack.
            \item If the addition is not nine, the read number must be placed on the first stack. Once finished the first treatment, you must empty the second stack number by number in a way that, when two or more consecutive numbers add up to nine or more, a nine must be placed on a third stack.
        \end{itemize}
        The final output of the program will be the number of nines contained in the third stack.


        \subsubsection{Developing ReadFile class}
        \subsubsection{Developing Problem class}
        \subsubsection{Developing Main class}
	
	
        
	
	\section{Queues}
	

	\section{Lists}

	

	

	
%	\newpage
	\bibliography{refs}
	
	
	
\end{document}